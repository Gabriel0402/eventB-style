% \iffalse meta-comment
% 
% eventB.ins
% 
% Copyright (C) 2012 by Thai Son Hoang <htson at inf dot ethz dot ch>
% --------------------------------------------------------------------
% 
% This file may be distributed and/or modified under the
% conditions of the LaTeX Project Public License, either version 1.3
% of this license or (at your option) any later version.
% The latest version of this license is in:
% 
%      http://www.latex-project.org/lppl.txt
% 
% and version 1.3 or later is part of all distributions of LaTeX 
% version 2003/12/01 or later.
% 
% This work has the LPPL maintenance status "author-maintained".
% 
% This work consists of the files eventB.dtx, eventB.ins,
% the derived file eventB.sty, the generated documentation
% eventB.pdf, and some sample documents.
% 
% \fi
% 
% \iffalse
%<eventB>%%%%% BEGIN Identification part %%%%%
%<eventB>\NeedsTeXFormat{LaTeX2e}\relax
%<eventB>\ProvidesPackage{eventB}
%<eventB>    [2012/02/19 v1.0.1 Package for typesetting Event-B models] 
%<eventB>%%%%% END Identification part %%%%%
% 
%<*driver> 
\documentclass[a4paper]{ltxdoc}
\EnableCrossrefs
% ^^A\CodelineIndex
\PageIndex
\RecordChanges

\begin{document}
\DocInput{eventB.dtx}
\end{document}
%</driver> 
% \fi
% 
% \CheckSum{0}
% 
% \CharacterTable
% {Upper-case    \A\B\C\D\E\F\G\H\I\J\K\L\M\N\O\P\Q\R\S\T\U\V\W\X\Y\Z
% Lower-case    \a\b\c\d\e\f\g\h\i\j\k\l\m\n\o\p\q\r\s\t\u\v\w\x\y\z
% Digits        \0\1\2\3\4\5\6\7\8\9
% Exclamation   \!     Double quote  \"     Hash (number) \#
% Dollar        \$     Percent       \%     Ampersand     \&
% Acute accent  \'     Left paren    \(     Right paren   \)
% Asterisk      \*     Plus          \+     Comma         \,
% Minus         \-     Point         \.     Solidus       \/
% Colon         \:     Semicolon     \;     Less than     \<
% Equals        \=     Greater than  \>     Question mark \?
% Commercial at \@     Left bracket  \[     Backslash     \\
% Right bracket \]     Circumflex    \^     Underscore    \_
% Grave accent  \`     Left brace    \{     Vertical bar  \|
% Right brace   \}     Tilde         \~}
% 
% 
% \changes{v1.0}{2012/07/03}{Initial version}
% \changes{v1.0.1}{2013/02/19}{Ensure that the keywords, labels are in math-mode}
% 
% \GetFileInfo{eventB.sty}
%
% \title{The \textsf{eventB} package\thanks{This document
% corresponds to \textsf{eventB}~\fileversion, dated~\filedate.}}
% \author{Thai Son Hoang \\ ETH-Zurich \\ \texttt{<htson at inf dot
% ethz dot ch>}}
% \date{July 3, 2012}
% 
% \maketitle
% 
% ^^A %%%%% Abstract %%%%%
% \begin{abstract}
%   This class provides a template for typesetting Event-B models.  It was
%   developed at the Swiss Federal Institute of Technology Zurich
%   (ETH-Zurich).
% \end{abstract}
% 
% ^^A %%%%% Table of contents %%%%%
% \tableofcontents
% 
% ^^A %%%%% Introduction %%%%%
% \section{Introduction}
% 
% This package was developed in order to ease the typesetting of
% Event-B models in \LaTeX{}.
% 
% ^^A %%%%% Usage %%%%%%
% \section{Usage}
% See |sample-eventB.tex| for an example of how to use the package.
% 
% \StopEventually{
% \PrintIndex
% \PrintChanges
% }
%
% ^^A %%%%% Implementation %%%%%
% \section{Implementation}
%
% \iffalse ^^A BEGIN Produce comments only in the resulting class file
%<eventB>
%<eventB>%%%%% BEGIN Declaration of options %%%%%
% \fi^^A END Produce comments only in the resulting class file
%    \begin{macrocode}
%%%%% BEGIN Package loading %%%%%
\RequirePackage{xspace}
\RequirePackage{xcolor}
%%%%% END Package loading %%%%%

%%%%% Keywords can be coloured by package option "color"
\newcommand{\Bkeyword}[1]{\ensuremath{\B@keyword{#1}}\xspace}
\newcommand{\Bidentifier}[1]{\B@identifier{#1}\xspace}
\newcommand{\Blabel}[2][]{\ensuremath{\B@label[#1]{#2}}\xspace}
\newcommand{\Bpo}[1]{\ensuremath{\B@po{#1}}\xspace}
\newcommand{\Bvspace}[1][2ex]{\\[#1]}
\newcommand{\Bhspace}[1][2em]{\hspace{#1}}
\newcommand{\Bsep}{\quad}
\DeclareOption{color}{
  \colorlet{Bkeywordcolor}{blue}
  \colorlet{Blabelcolor}{green!50!black}
  \colorlet{Bpocolor}{red}
  %%%%% Keywords are coloured.
  \renewcommand{\Bkeyword}[1]{\ensuremath{\textcolor{Bkeywordcolor}{\B@keyword{#1}}}\xspace}
  \renewcommand{\Blabel}[2][]{\ensuremath{\textcolor{Blabelcolor}{\B@label[#1]{#2}}}\xspace}
  \renewcommand{\Bpo}[1]{\ensuremath{\textcolor{Bpocolor}{\B@po{#1}}}\xspace}
}
\DeclareOption{colour}{
  \colorlet{Bkeywordcolor}{blue}
  \colorlet{Blabelcolor}{green!50!black}
  \colorlet{Bpocolor}{red}
  %%%%% Keywords are coloured.
  \renewcommand{\Bkeyword}[1]{\ensuremath{\textcolor{Bkeywordcolor}{\B@keyword{#1}}}\xspace}
  \renewcommand{\Blabel}[2][]{\ensuremath{\textcolor{Blabelcolor}{\B@label[#1]{#2}}}\xspace}
  \renewcommand{\Bpo}[1]{\ensuremath{\textcolor{Bpocolor}{\B@po{#1}}}\xspace}
}

\newcommand{\B@keyword}[1]{\ensuremath{\mathbf{#1}}\xspace}
\newcommand{\B@identifier}[1]{\ensuremath{\mathit{#1}}\xspace}
\newcommand{\B@label}[2][]{
  \def\is@thm{#1}
  \ifx\is@thm\@empty
  \ensuremath{\mathsf{#2}}\xspace
  \else
  \ensuremath{\mathit{#2}}\xspace
  \fi
}

%%%%% Bcode environment %%%%%
%%%%% the same as ``center'' %%%%%
\newenvironment{Bcode}[1][\normalsize]{\begin{center}#1}{\end{center}}
\newcommand{\Bdeclaration}[2]{
  \fbox{
    \ensuremath{
      \B@declaration{#1}{#2}
    }
  }
}

\newcommand{\Bsection}[3][]{
  \setlength{\B@oldfboxsep}{\fboxsep}
  \setlength{\fboxsep}{2ex}
  \fbox{
    \ensuremath{
      \B@section[#1]{#2}{#3}
    }
  }
  \setlength{\fboxsep}{\B@oldfboxsep}
}

\newcommand{\event}[7][]{
  \setlength{\B@oldfboxsep}{\fboxsep}
  \setlength{\fboxsep}{2ex}
  \fbox{
    \ensuremath{
      \B@event[#1]{#2}{#3}{#4}{#5}{#6}{#7}
    }
  }
  \setlength{\fboxsep}{\B@oldfboxsep}
}

\newcommand{\eventinline}[7][]{
  \setlength{\B@oldfboxsep}{\fboxsep}
  \setlength{\fboxsep}{2ex}
  \fbox{
    \ensuremath{
      \B@eventinline[#1]{#2}{#3}{#4}{#5}{#6}{#7}
    }
  }
  \setlength{\fboxsep}{\B@oldfboxsep}
}

\DeclareOption{compact}{
  \renewenvironment{Bcode}[1][\footnotesize]{\begin{center}#1}{\end{center}}
  \renewcommand{\Bdeclaration}[2]{
    \B@declaration{#1}{#2}
  }

  \renewcommand{\Bsection}[3][]{
    \B@section[#1]{#2}{#3}
  }

  \renewcommand{\event}[7][]{
    \B@event[#1]{#2}{#3}{#4}{#5}{#6}{#7}
  }
  
  \renewcommand{\eventinline}[7][]{
    \B@eventinline[#1]{#2}{#3}{#4}{#5}{#6}{#7}
  }

  \renewcommand{\Bvspace}[1][0ex]{\\[#1]}
  \renewcommand{\Bhspace}[1][0.5em]{\hspace{#1}}
  \renewcommand{\Bsep}{\ }
}
\DeclareOption{small}{
  \renewenvironment{Bcode}[1][\footnotesize]{\begin{center}#1}{\end{center}}
  \renewcommand{\Bvspace}[1][1ex]{\\[#1]}
  \renewcommand{\Bhspace}[1][1em]{\hspace{#1}}
  \renewcommand{\Bdeclaration}[2]{
    \B@declaration{#1}{#2}
  }

  \renewcommand{\Bsection}[3][]{
    \B@section[#1]{#2}{#3}
  }

  \renewcommand{\event}[7][]{
    \B@event[#1]{#2}{#3}{#4}{#5}{#6}{#7}
  }
  \renewcommand{\eventinline}[7][]{
    \B@eventinline[#1]{#2}{#3}{#4}{#5}{#6}{#7}
  }

  \renewcommand{\Bsep}{\ }

}
\DeclareOption{tiny}{
  \renewenvironment{Bcode}[1][\scriptsize]{\begin{center}#1}{\end{center}}
  \renewcommand{\Bvspace}[1][-0.5ex]{\\[#1]}
  \renewcommand{\Bhspace}[1][0.5em]{\hspace{#1}}
  \renewcommand{\Bdeclaration}[2]{
    \B@declaration{#1}{#2}
  }

  \renewcommand{\Bsection}[3][]{
    \B@section[#1]{#2}{#3}
  }

  \renewcommand{\event}[7][]{
    \B@event[#1]{#2}{#3}{#4}{#5}{#6}{#7}
  }
  \renewcommand{\eventinline}[7][]{
    \B@eventinline[#1]{#2}{#3}{#4}{#5}{#6}{#7}
  }

  \renewcommand{\Bsep}{\ }

}

\newcommand{\B@declaration}[2]{
  \begin{array}{l@{\Bsep}l}
    \Bkeyword{#1:} &  #2 
  \end{array}
}
\newlength{\B@oldfboxsep}
\newcommand{\B@section}[3][]{
  \def\no@title{#1}
  \ifx\no@title\@empty
  \begin{array}{l}
    \Bkeyword{#2:} \\
    \begin{array}{l@{\Bsep}l}
      #3
    \end{array}
  \end{array}
  \else
  \begin{array}{l@{\Bsep}l}
    #3
  \end{array}
  \fi
}

\newcommand{\B@po}[1]{\ensuremath{\mathsf{#1}}\xspace}

%    \end{macrocode}
% \iffalse ^^A BEGIN Produce comments only in the resulting class file
%<eventB>%%%%% END Declaration of options %%%%%
%<eventB>
% \fi^^A END Produce comments only in the resulting class file
%
%
%    \begin{macrocode}
%%%%% BEGIN Execution of options %%%%%
\ProcessOptions
%%%%% END Execution of options %%%%%

%    \end{macrocode}
%
%
%
%    \begin{macrocode}
%%%%% (BEGIN) Macros for Pretty-Print Event-B Components %%%
\newcommand{\eventB}{Event-B\xspace}
\newcommand{\SKIP}{\textsc{skip}}

%%%%% Event-B Keywords %%%%%
\newcommand{\Bany}{\Bkeyword{any}}
\newcommand{\Bbegin}{\Bkeyword{begin}}
\newcommand{\Bend}{\Bkeyword{end}}
\newcommand{\Brefines}{\Bkeyword{refines}}
\newcommand{\Bstatus}{\Bkeyword{status}}
\newcommand{\Bthen}{\Bkeyword{then}}
\newcommand{\Bwhen}{\Bkeyword{when}}
\newcommand{\Bwhere}{\Bkeyword{where}}
\newcommand{\Bwith}{\Bkeyword{with}}

%    \end{macrocode}
%
%
% 
%%%%%    \begin{macrocode}
%%%%% Event-B internal elements %%%%%
\newcommand{\Bctx}[1]{\ensuremath{\mathbf{#1}}\xspace}
\newcommand{\Bset}[1]{\Bidentifier{#1}}
\newcommand{\Bcst}[1]{\Bidentifier{#1}}
\newcommand{\Baxm}[1]{\Blabel{#1}}
\newcommand{\Bthm}[1]{\Blabel[thm]{#1}}

\newcommand{\Bmch}[1]{\ensuremath{\mathbf{#1}}\xspace}
\newcommand{\Bvrb}[1]{\Bidentifier{#1}}
\newcommand{\Binv}[1]{\Blabel{#1}}
\newcommand{\Bevt}[1]{\Blabel{#1}}
\newcommand{\Bpar}[1]{\Bidentifier{#1}}
\newcommand{\Bact}[1]{\Blabel{#1}}
\newcommand{\Bgrd}[1]{\Blabel{#1}}
\newcommand{\Bbap}[1]{\hbox{\sl\bfseries #1}}
%%%%%    \end{macrocode}
%
%

%    \begin{macrocode}
%%%%% Creating Event-B elements macros %%%%%

%%%%% Create a new B macro
%%%%% Arguments:
%%%%% 1. The macro string, (OPTIONAL) if empty then the expanded string will be used.
%%%%% 2. The expanded string
%%%%% 3. The mark-up macros, e.g. \Bvrb
%%%%% Usage:
%%%%% - \B@newmacro[aaa]{a\_a\_a}{\Bvrb} will create a new macro \aaa
%%%%% which will be expanded to be \Bvrb{a\_a\_a}
%%%%% - \B@newmacro{aaa}{\Bvrb} will create a new macro \aaa
%%%%% which will be expanded to be \Bvrb{aaa}
%%%%%
\newcommand{\B@newmacro}[3][]{
  \def\input@macro{#1}
  \ifx\input@macro\@empty
  \expandafter\def\csname #2\endcsname{#3{#2}}
  \else
  \expandafter\def\csname #1\endcsname{#3{#2}}
  \fi
}

%    \end{macrocode}
%
%
%    \begin{macrocode}
%%%%% Create a new context macro
%%%%% Arguments:
%%%%% 1. The macro string (OPTIONAL)
%%%%% 2. The expanded string
%%%%% Usage:
%%%%% - \newBctx[aaa]{a\_a\_a} will create a new macro \aaa
%%%%% which will be expanded to be \Bctx{a\_a\_a}.
%%%%% - \newBctx{aaa} will create a new macro \aaa which will be
%%%%% expanded to be \Bctx{aaa}.
%%%%%
\newcommand{\newBctx}[2][]{%
  \B@newmacro[#1]{#2}{\Bctx}
}

%    \end{macrocode}

%    \begin{macrocode}
%%%%% Create a new carrier set macro
%%%%% Arguments:
%%%%% 1. The macro string (OPTIONAL)
%%%%% 2. The expanded string
%%%%% Usage:
%%%%% - \newBset[aaa]{a\_a\_a} will create a new macro \aaa
%%%%% which will be expanded to be \Bset{a\_a\_a}.
%%%%% - \newBset{aaa} will create a new macro \aaa which will be
%%%%% expanded to be \Bset{aaa}.
%%%%%
\newcommand{\newBset}[2][]{%
  \B@newmacro[#1]{#2}{\Bset}
}

%    \end{macrocode}

%    \begin{macrocode}
%%%%% Create a new constant macro
%%%%% Arguments:
%%%%% 1. The macro string (OPTIONAL)
%%%%% 2. The expanded string
%%%%% Usage:
%%%%% - \newBcst[aaa]{a\_a\_a} will create a new macro \aaa
%%%%% which will be expanded to be \Bcst{a\_a\_a}.
%%%%% - \newBcst{aaa} will create a new macro \aaa which will be
%%%%% expanded to be \Bcst{aaa}.
%%%%%
\newcommand{\newBcst}[2][]{%
  \B@newmacro[#1]{#2}{\Bcst}
}

%    \end{macrocode}

%    \begin{macrocode}
%%%%% Create a new axiom macro
%%%%% Arguments:
%%%%% 1. The macro string (OPTIONAL)
%%%%% 2. The expanded string
%%%%% Usage:
%%%%% - \newBaxm[aaa]{a\_a\_a} will create a new macro \aaa
%%%%% which will be expanded to be \Baxm{a\_a\_a}.
%%%%% - \newBaxm{aaa} will create a new macro \aaa which will be
%%%%% expanded to be \Baxm{aaa}.
%%%%%
\newcommand{\newBaxm}[2][]{%
  \B@newmacro[#1]{#2}{\Baxm}
}

%    \end{macrocode}

%    \begin{macrocode}
%%%%% Create a new theorem macro
%%%%% Arguments:
%%%%% 1. The macro string (OPTIONAL)
%%%%% 2. The expanded string
%%%%% Usage:
%%%%% - \newBthm[aaa]{a\_a\_a} will create a new macro \aaa
%%%%% which will be expanded to be \Bthm{a\_a\_a}.
%%%%% - \newBthm{aaa} will create a new macro \aaa which will be
%%%%% expanded to be \Bthm{aaa}.
%%%%%
\newcommand{\newBthm}[2][]{%
  \B@newmacro[#1]{#2}{\Bthm}
}

%    \end{macrocode}

%    \begin{macrocode}
%%%%% Create a new machine macro
%%%%% Arguments:
%%%%% 1. The macro string (OPTIONAL)
%%%%% 2. The expanded string
%%%%% Usage:
%%%%% - \newBmch[aaa]{a\_a\_a} will create a new macro \aaa
%%%%% which will be expanded to be \Bmch{a\_a\_a}.
%%%%% - \newBmch{aaa} will create a new macro \aaa which will be
%%%%% expanded to be \Bmch{aaa}.
%%%%%
\newcommand{\newBmch}[2][]{%
  \B@newmacro[#1]{#2}{\Bmch}
}

%    \end{macrocode}
%
%    \begin{macrocode}
%%%%% Create a new variable macro
%%%%% Arguments:
%%%%% 1. The macro string (OPTIONAL)
%%%%% 2. The expanded string
%%%%% Usage:
%%%%% - \newBvrb[aaa]{a\_a\_a} will create a new macro \aaa
%%%%% which will be expanded to be \Bvrb{a\_a\_a}.
%%%%% - \newBvrb{aaa} will create a new macro \aaa which will be
%%%%% expanded to be \Bvrb{aaa}.
%%%%%
\newcommand{\newBvrb}[2][]{%
  \B@newmacro[#1]{#2}{\Bvrb}
}

%    \end{macrocode}

%    \begin{macrocode}
%%%%% Create a new invariant macro
%%%%% Arguments:
%%%%% 1. The macro string (OPTIONAL)
%%%%% 2. The expanded string
%%%%% Usage:
%%%%% - \newBinv[aaa]{a\_a\_a} will create a new macro \aaa
%%%%% which will be expanded to be \Binv{a\_a\_a}.
%%%%% - \newBinv{aaa} will create a new macro \aaa which will be
%%%%% expanded to be \Binv{aaa}.
%%%%%
\newcommand{\newBinv}[2][]{%
  \B@newmacro[#1]{#2}{\Binv}
}

%    \end{macrocode}

%    \begin{macrocode}
%%%%% Create a new event macro
%%%%% Arguments:
%%%%% 1. The macro string (OPTIONAL)
%%%%% 2. The expanded string
%%%%% Usage:
%%%%% - \newBevt[aaa]{a\_a\_a} will create a new macro \aaa
%%%%% which will be expanded to be \Bevt{a\_a\_a}.
%%%%% - \newBevt{aaa} will create a new macro \aaa which will be
%%%%% expanded to be \Bevt{aaa}.
%%%%%
\newcommand{\newBevt}[2][]{%
  \B@newmacro[#1]{#2}{\Bevt}
}

%    \end{macrocode}

%    \begin{macrocode}
%%%%% Create a new parameter macro
%%%%% Arguments:
%%%%% 1. The macro string (OPTIONAL)
%%%%% 2. The expanded string
%%%%% Usage:
%%%%% - \newBpar[aaa]{a\_a\_a} will create a new macro \aaa
%%%%% which will be expanded to be \Bpar{a\_a\_a}.
%%%%% - \newBpar{aaa} will create a new macro \aaa which will be
%%%%% expanded to be \Bpar{aaa}.
%%%%%
\newcommand{\newBpar}[2][]{%
  \B@newmacro[#1]{#2}{\Bpar}
}

%    \end{macrocode}
%

%    \begin{macrocode}
%%%%% Create a new guard macro
%%%%% Arguments:
%%%%% 1. The macro string (OPTIONAL)
%%%%% 2. The expanded string
%%%%% Usage:
%%%%% - \newBgrd[aaa]{a\_a\_a} will create a new macro \aaa
%%%%% which will be expanded to be \Bgrd{a\_a\_a}.
%%%%% - \newBgrd{aaa} will create a new macro \aaa which will be
%%%%% expanded to be \Bgrd{aaa}.
%%%%%
\newcommand{\newBgrd}[2][]{%
  \B@newmacro[#1]{#2}{\Bgrd}
}

%    \end{macrocode}
%
%    \begin{macrocode}
%%%%% Create a new action macro
%%%%% Arguments:
%%%%% 1. The macro string (OPTIONAL)
%%%%% 2. The expanded string
%%%%% Usage:
%%%%% - \newBact[aaa]{a\_a\_a} will create a new macro \aaa
%%%%% which will be expanded to be \Bact{a\_a\_a}.
%%%%% - \newBact{aaa} will create a new macro \aaa which will be
%%%%% expanded to be \Bact{aaa}.
%%%%%
\newcommand{\newBact}[2][]{%
  \B@newmacro[#1]{#2}{\Bact}
}

%    \end{macrocode}
%

%
%    \begin{macrocode}
%%%%% Pretty print carrier sets
%%%%% Arguments:
%%%%% 1. (Comma-separated) list of carrier sets.
%%%%%
%%%%% Usage: \carriersets{S, T}
\newcommand{\carriersets}[1]{
  \Bdeclaration{sets}{#1}
}

%%%%% Pretty print constants
%%%%% Arguments:
%%%%% 1. (Comma-separated) list of constants.
%%%%%
%%%%% Usage: \constants{m, n}
\newcommand{\constants}[1]{
  \Bdeclaration{constants}{#1}
}

%    \end{macrocode}

%    \begin{macrocode}
%%%%% Pretty print axioms
%%%%% Arguments:
%%%%% 1. (Newline(\\)-separated) list of axioms.
%%%%%
%%%%% Usage: \axioms{\Baxm{axm0\_1}: & x \in \nat \\
%%%%%                \Baxm{axm0\_2}: & y \in \nat \\[2ex]}
\newcommand{\axioms}[2][]{
  \Bsection[#1]{axioms}{#2}
}

%    \end{macrocode}
%
%
%    \begin{macrocode}
%%%%% Pretty print variables
%%%%% Arguments:
%%%%% 1. (Comma-separated) list of variables.
%%%%%
%%%%% Usage: \variables{x, y}
\newcommand{\variables}[1]{
  \Bdeclaration{variables}{#1}
}

%    \end{macrocode}
%
%
%    \begin{macrocode}
%%%%% Pretty print invariants
%%%%% Arguments:
%%%%% 1. (Newline(\\)-separated) list of invariants.
%%%%%
%%%%% Usage: \invariants{\Binv{inv0\_1:} & x \in \nat \\
%%%%%                    \Binv{inv0\_2:} & y \in \nat \\[2ex]}
\newcommand{\invariants}[2][]{
  \Bsection[#1]{invariants}{#2}
}

%    \end{macrocode}

%
%
%    \begin{macrocode}
%%%%% Pretty print variant
%%%%% Arguments:
%%%%% 1. The variant
%%%%%
%%%%% Usage: \variant{V}
\newcommand{\variant}[1]{
  \Bdeclaration{variant}{#1}
}

%    \end{macrocode}
%
%
%
%
%    \begin{macrocode}
%%%%% Pretty print an general Event-B event
%%%%% Arguments:
%%%%% 1. (Optional) convergence status.
%%%%% 2. Name of the event.
%%%%% 3. Name of the abstract event.
%%%%% 4. (Comma-separated) list of parameters.
%%%%% 5. (Newline(\\)-separated) list of guards.
%%%%% 6. (Newline(\\)-separated) list of witness predicates.
%%%%% 7. (Newline(\\)-separated) list of assignments.
%%%%%
%%%%% Usage: \B@event[conv]{conc}{abs}{x,y}{G1(x,y)\\G2(x,y)}{W1\\W2}{S1(v,x,y)\\S2(w,x,y)}
%%%%%       will produce the following
%%%%%
%%%%%       conc
%%%%%       refines abs
%%%%%       status conv
%%%%%       any x, y where
%%%%%         G1(x, y)
%%%%%         G2(x, y)
%%%%%       with
%%%%%         W1
%%%%%         W2
%%%%%       then
%%%%%         S1(v, x, y)
%%%%%         S2(w, x, y)
%%%%%       end
%%%%%
%%%%% Special case:
%%%%% - Empty abstract event --> refines clause is omitted.
%%%%% - Empty convergence status --> status clause is omitted.
%%%%% - Empty witness --> with clause is omitted.
%%%%% - Empty parameters, empty guards --> begin ... end
%%%%% - Empty parameters --> when ... then ... end
%%%%% - Empty actions --> \SKIP
\newcommand{\B@event}[7][]{
  \def\evt@sts{#1}
  \def\evt@name{#2}
  \def\evt@absevts{#3}
  \def\evt@pars{#4}
  \def\evt@grds{#5}
  \def\evt@wits{#6}
  \def\evt@acts{#7}
  %% Pretty-print convergence status
  \ifx\evt@sts\@empty
  \def\pretty@sts{}
  \else
  \def\pretty@sts{\Bsep\Bstatus \Bsep \evt@sts \\}
  \fi
  % Pretty-print abstract events
  \ifx\evt@absevts\@empty
  \def\pretty@absevts{}
  \else
  \def\pretty@absevts{\Bsep\Brefines \Bsep \evt@absevts \\}
  \fi
  % Pretty-print parameters
  \ifx\evt@pars\@empty
  \def\pretty@pars{}
  \else
  \def\pretty@pars{\Bsep\Bany \Bsep \evt@pars \Bsep \Bwhere \\}
  \fi
  % Pretty-print guards
  \ifx\evt@grds\@empty
  \def\pretty@grds{}
  \else
  \def\evt@grds@tmp{
    \begin{array}{@{\Bsep\Bsep}l@{\Bsep}l}
      \evt@grds
    \end{array}\\
  }
  \ifx\evt@pars\@empty
  \def\pretty@grds{
    \Bsep \Bwhen \\
    \evt@grds@tmp
  }
  \else
  \def\pretty@grds{\evt@grds@tmp}
  \fi
  \fi
  % Pretty-print witnesses
  \ifx\evt@wits\@empty
  \def\pretty@wits{}
  \else
  \def\pretty@wits{
    \Bsep\Bwith\\
    \begin{array}{@{\Bsep\Bsep}ll}
      \evt@wits
    \end{array}\\
  }
  \fi
  % Pretty-print actions
  \ifx\evt@acts\@empty
  \def\evt@acts{\SKIP}
  \else
  \fi  
  \def\evt@acts@tmp{
    \begin{array}{@{\Bsep\Bsep}l@{\Bsep}l}
      \evt@acts
    \end{array}\\
  }
  \def\evt@acts@keyword{\Bsep\Bthen \\}
  \ifx\evt@pars\@empty
  \ifx\evt@grds\@empty
  \def\evt@acts@keyword{\Bsep\Bbegin \\}
  \else
  \fi
  \else
  \fi
  \def\pretty@acts{
    \evt@acts@keyword
    \evt@acts@tmp
  }
  % Really do it now
  \begin{array}{l}
    \Bevt{\evt@name} \\
    \pretty@sts
    \pretty@absevts
    \pretty@pars
    \pretty@grds
    \pretty@wits
    \pretty@acts
    \Bsep\Bend
  \end{array}
}

%    \end{macrocode}
%
%
%    \begin{macrocode}
%%%%% Pretty print an general Event-B event
%%%%% Arguments:
%%%%% 1. (Optional) convergence status.
%%%%% 2. Name of the event.
%%%%% 3. Name of the abstract event.
%%%%% 4. (Comma-separated) list of parameters.
%%%%% 5. (Newline(\\)-separated) list of guards.
%%%%% 6. (Newline(\\)-separated) list of witness predicates.
%%%%% 7. (Newline(\\)-separated) list of assignments.
%%%%%
%%%%% Usage: \B@event[conv]{conc}{abs}{x,y}{G1(x,y)\\G2(x,y)}{W1\\W2}{S1(v,x,y)\\S2(w,x,y)}
%%%%%       will produce the following
%%%%%
%%%%%       conc
%%%%%       refines abs
%%%%%       status conv
%%%%%       any x, y where
%%%%%         G1(x, y)
%%%%%         G2(x, y)
%%%%%       with
%%%%%         W1
%%%%%         W2
%%%%%       then
%%%%%         S1(v, x, y)
%%%%%         S2(w, x, y)
%%%%%       end
%%%%%
%%%%% Special case:
%%%%% - Empty abstract event --> refines clause is omitted.
%%%%% - Empty convergence status --> status clause is omitted.
%%%%% - Empty witness --> with clause is omitted.
%%%%% - Empty parameters, empty guards --> begin ... end
%%%%% - Empty parameters --> when ... then ... end
%%%%% - Empty actions --> \SKIP
\newcommand{\B@eventinline}[7][]{
  \def\evt@sts{#1}
  \def\evt@name{#2}
  \def\evt@absevts{#3}
  \def\evt@pars{#4}
  \def\evt@grds{#5}
  \def\evt@wits{#6}
  \def\evt@acts{#7}
  %% Ignore convergence status
  \def\pretty@sts{}
  % Pretty-print abstract events
  \ifx\evt@absevts\@empty
  \def\pretty@absevts{}
  \else
  \def\pretty@absevts{\Brefines~\evt@absevts~}
  \fi
  % Pretty-print parameters
  \ifx\evt@pars\@empty
  \def\pretty@pars{}
  \else
  \def\pretty@pars{\Bany~\evt@pars~\Bwhere~}
  \fi
  % Pretty-print guards
  \ifx\evt@grds\@empty
  \def\pretty@grds{}
  \else
  \def\evt@grds@tmp{
      \evt@grds
  }
  \ifx\evt@pars\@empty
  \def\pretty@grds{
    \Bwhen~
    \evt@grds@tmp~
  }
  \else
  \def\pretty@grds{\evt@grds@tmp~}
  \fi
  \fi
  % Pretty-print witnesses
  \ifx\evt@wits\@empty
  \def\pretty@wits{}
  \else
  \def\pretty@wits{
    \Bwith~
    \evt@wits~
  }
  \fi
  % Pretty-print actions
  \ifx\evt@acts\@empty
  \def\evt@acts{\SKIP}
  \else
  \fi  
  \def\evt@acts@tmp{
    \evt@acts
  }
  \def\evt@acts@keyword{\Bthen}
  \ifx\evt@pars\@empty
  \ifx\evt@grds\@empty
  \def\evt@acts@keyword{\Bbegin}
  \else
  \fi
  \else
  \fi
  \def\pretty@acts{
    \evt@acts@keyword~
    \evt@acts@tmp~
  }
  % Really do it now
  \begin{array}{l}
    \Bevt{\evt@name}~\widehat{=}~
    \pretty@sts
    \pretty@absevts
    \pretty@pars
    \pretty@grds
    \pretty@wits
    \pretty@acts
    \Bend
  \end{array}
}

%    \end{macrocode}
%
%    \begin{macrocode}
%%%%% INITIALISATION label
\newBevt{init}

%%%%% Pretty print the initialisation: no ``refines'' clause. no parameters, no
%%%%% guards
%%%%% Arguments:
%%%%% 1. (Newline(\\)-separated) list of assignments.
%%%%%
%%%%% Usage: \init{S1(v,x,y)\\S2(w,x,y)}
%%%%%       will produce the following
%%%%%
%%%%%       init
%%%%%       begin
%%%%%         S1(v, x, y)
%%%%%         S2(w, x, y)
%%%%%       end
%%%%%
\newcommand{\initialisation}[1]{
  \event{\init}{}{}{}{}{#1}
}
%    \end{macrocode}
%
%
%
%    \begin{macrocode}

%%%%% Theorem Proof Obligation
%%%%% Print the theorem proof obligation, given the theorem label.
%%%%% Arguments:
%%%%% 1. Theorem label
%%%%%
%%%%% Usage:
%%%%% - \thmpo{thm} will produce "thm/THM"
\newcommand{\thmpo}[1]{\Bthm{#1}/\Bpo{THM}}

%    \end{macrocode}
%
%    \begin{macrocode}
%%%%% Axiom Well-definedness Proof Obligation
%%%%% Print the axiom well-definedness proof obligation, given the
%%%%% axiom label.
%%%%% Arguments:
%%%%% 1. Axiom label
%%%%%
%%%%% Usage:
%%%%% - \axmwdpo{axm} will produce "axm/WD"
\newcommand{\axmwdpo}[1]{\Baxm{#1}/\Bpo{WD}}

%    \end{macrocode}
%
%    \begin{macrocode}
%%%%% Invariant Proof Obligation
%%%%% Print the invariant proof obligation, given the event name and
%%%%% invariant label
%%%%% Arguments:
%%%%% 1. Event name
%%%%% 2. Invariant label
%%%%%
%%%%% Usage:
%%%%% - \invpo{evt}{inv} will produce "evt/inv/INV"
\newcommand{\invpo}[2]{\Bevt{#1}/\Binv{#2}/\Bpo{INV}}

%    \end{macrocode}
%
%
%    \begin{macrocode}
%%%%% Theorem (in guard) Proof Obligation
%%%%% Print the simulation proof obligation, given the event name and
%%%%% the theorem (in guard) label.
%%%%% Arguments:
%%%%% 1. Event name
%%%%% 2. Theorem (in guard) label
%%%%%
%%%%% Usage:
%%%%% - \grdthmpo{evt}{thm} will produce "evt/thm/THM"
\newcommand{\grdthmpo}[2]{\Bevt{#1}/\Bthm{#2}/\Bpo{THM}}

%    \end{macrocode}
%
%
%    \begin{macrocode}
%%%%% Feasibility Proof Obligation
%%%%% Print the feasibility proof obligation, given the event name and
%%%%% the action label
%%%%% Arguments:
%%%%% 1. Event name
%%%%% 2. Action label
%%%%%
%%%%% Usage:
%%%%% - \fispo{evt}{act} will produce "evt/act/FIS"
\newcommand{\fispo}[2]{\Bevt{#1}/\Bact{#2}/\Bpo{FIS}}

%    \end{macrocode}
%

%    \begin{macrocode}
%%%%% Variant finiteness Proof Obligation
%%%%% Print the Variant finiteness proof obligation
%%%%% Arguments: No arguments
%%%%%
%%%%% Usage:
%%%%% - \finpo will produce "FIN"
\newcommand{\finpo}{\Bpo{FIN}}

%    \end{macrocode}
%
%
%
%    \begin{macrocode}
%%%%% Variant Proof Obligation
%%%%% Print the guard strengthen proof obligation, given the event name
%%%%% Arguments:
%%%%% 1. Event name
%%%%%
%%%%% Usage:
%%%%% - \grdpo{evt} will produce "evt/VAR"
\newcommand{\varpo}[1]{\Bevt{#1}/\Bpo{VAR}}

%    \end{macrocode}
%
%
%
%    \begin{macrocode}
%%%%% Simulation Proof Obligation
%%%%% Print the simulation proof obligation, given the event name and
%%%%% the action label.
%%%%% Arguments:
%%%%% 1. Event name
%%%%% 2. Action label
%%%%%
%%%%% Usage:
%%%%% - \simpo{evt}{act} will produce "evt/act/SIM"
\newcommand{\simpo}[2]{\Bevt{#1}/\Bact{#2}/\Bpo{SIM}}

%    \end{macrocode}
%
%    \begin{macrocode}
%%%%% Guard Strengthen Proof Obligation
%%%%% Print the guard strengthen proof obligation, given the event
%%%%% name and the guard label
%%%%% Arguments:
%%%%% 1. (Abstract) Event name
%%%%% 2. (Abstract) Guard label
%%%%%
%%%%% Usage:
%%%%% - \grdpo{evt}{grd} will produce "evt/grd/GRD"
\newcommand{\grdpo}[2]{\Bevt{#1}/\Bgrd{#2}/\Bpo{GRD}}

%    \end{macrocode}
%
%
%
%    \begin{macrocode}
%%%%% Variant Natural Number Proof Obligation
%%%%% Print the Variant Natural Number proof obligation, given the event name
%%%%% Arguments:
%%%%% 1. Event name
%%%%%
%%%%% Usage:
%%%%% - \natpo{evt} will produce "evt/NAT"
\newcommand{\natpo}[1]{\Bevt{#1}/\Bpo{NAT}}

%    \end{macrocode}
% \Finale
\endinput

