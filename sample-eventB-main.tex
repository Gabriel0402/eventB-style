\usepackage{bsymb}

\title{A Sample Event-B Model}
\author{Thai Son Hoang}







\newBmch[datarefMch]{m2}
\newBvrb[attendees]{atnds}
\newBevt{OpenCourse}
\newBevt{CloseCourse}


\begin{document}

\maketitle

This document provide a sample Event-B model using typeset using the
\textsf{eventB} package.

\section{Contexts}
\label{sec:contexts}

\newBctx{coursesCtx}
\newBset[COURSE]{CRS}
\newBcst[maxC]{m}
\newBaxm[axmFinCrs]{axm0\_1}
\newBaxm[axmMaxNonNeg]{axm0\_2}
\newBthm[thmMaxGrtZro]{thm0\_1}
\newBaxm[axmMaxCrdCrs]{axm0\_3}

Our initial context $\coursesCtx$ contains a carrier set $\COURSE$
denoting the set of courses that can be offered by the club.
Moreover, $\coursesCtx$ includes a constant $\maxC$ denoting the
maximum number of courses that the club can have at the same time.
The context \coursesCtx is as follows.
\begin{Bcode}
  $
  \carriersets{\COURSE}
  $
  \Bhspace
  $
  \constants{\maxC}
  $
  \Bvspace
  $
  \axioms{
    \axmFinCrs: & \finite(\COURSE) \\
    \axmMaxNonNeg: & \maxC \in \nat1 \\
    \thmMaxGrtZro: & 0 < \maxC \\
    \axmMaxCrdCrs: & \maxC \leq \card(\COURSE)
  }
  $
\end{Bcode}

Note that we label the axioms and theorems with the prefixes denoting
the role of the modelling elements, i.e., $\Baxm{axm}$ and $\Bthm{thm}$,
with some numbers.  For example, $\axmFinCrs$ denotes the first
(i.e., $1$) axiom for the initial model (i.e., $0$).  We apply this
systematic labelling through out our development.


The assumption on $\COURSE$ and $\maxC$ are captured by the axioms and
theorems as follows.  Axiom $\axmFinCrs$ states that $\COURSE$ is
finite.  Axiom $\axmMaxNonNeg$ states that $\maxC$ is a member of the
set of natural numbers (i.e., $\maxC$ is a natural number). Finally,
$\axmMaxNonNeg$ states that $\maxC$ cannot exceed the number of
possible courses that can be offered by the club, represented as
$\card(\COURSE)$, the cardinality of $\COURSE$.  A derived property of
$\maxC$ is presented as theorem $\thmMaxGrtZro$.

\paragraph{$\thmpo{\thmMaxGrtZro}$}
A proof obligation is generated for $\thmMaxGrtZro$ as follows.
Notice that $\axmMaxCrdCrs$ does not appear in the set of hypotheses
for the obligation, since it is declared after $\thmMaxGrtZro$.  By
convention, each proof obligation is labelled according to the
element involved and the name of the proof obligation rule.  Here
$\thmpo{\thmMaxGrtZro}$ indicates that it is a \Bpo{THM} proof obligation
for $\thmMaxGrtZro$.
\begin{Bcode}
  $\ldots$
\end{Bcode}
The obligations can be trivially discharged since $\nat1$ is the set of
all positive natural numbers, i.e., $\{1, 2, \ldots\}$.

\paragraph{$\axmwdpo{axm0\_3}$}
It is required to prove that $\axmMaxCrdCrs$ is well-defined.  The
corresponding proof obligation is as follows.
\begin{Bcode}
  $\ldots$
\end{Bcode}
Since the goal appears amongst the hypotheses, the proof obligation
can be discharged trivially.  Note that the order of appearance of the
axioms is important. In particular, $\axmMaxNonNeg$ needs to be
declared before $\axmMaxCrdCrs$.


\section{Machines}
\label{sec:machines}

\newBmch[coursesMch]{m0}
\newBvrb[courses]{crs}
\newBinv[invCrsTyp]{inv0\_1}
\newBthm[thmFinCrs]{thm0\_2}
\newBinv[invCrsLim]{inv0\_2}
\newBevt{OpenCourses}
\newBpar[course]{c}
\newBgrd[grdCrsLim]{grd0\_1}
\newBthm[thmCrsBelowLim]{thm0\_3}
\newBact[actOPCrsUpd]{act0\_1}
\newBevt{CloseCourses}
\newBpar{cs}
\newBthm[thmDlf]{DLF}

We develop machine $\coursesMch$ of the initial model, focusing on
courses opening and closing.  This machine sees context $\coursesCtx$
as developed in Section~\ref{sec:contexts}, hence as a result has
access to the carrier set $\COURSE$ and constant $\maxC$.  We model
the set of opened courses by a variable, namely $\courses$.  Invariant
$\Binv{inv0\_1}$ states that it is a subset of available courses
$\COURSE$.  A consequence of this invariant and of axiom
$\Baxm{axm0\_1}$ is that $\courses$ is finite, and this is stated in
$\coursesMch$ as theorem $\Bthm{thm0\_2}$.  invariant $\Binv{inv0\_2}$
states that the number of opened courses, i.e., $\card(\courses)$ is
bounded above by $\maxC$.  Initially, all courses are closed hence
$\courses$ is set to the empty set ($\emptyset$).
\begin{Bcode}
  $
  \variables{\courses}
  $
  \Bhspace
  $
  \invariants{
    \invCrsTyp: & \courses \subseteq \COURSE \\
    \thmFinCrs: & \finite(\courses) \\
    \invCrsLim: & \card(\courses) \leq \maxC
  }
  $
  \Bvspace
  $
  \initialisation{
    \courses \bcmeq \emptyset
  }
  $
\end{Bcode}

We model the opening and closing of courses using two events
$\OpenCourses$ and $\CloseCourses$ as follows.
\begin{Bcode}
  $
  \event[ordinary]{\OpenCourses}{}{}{
    \grdCrsLim: & \card(\courses) \neq \maxC \\
    \thmCrsBelowLim: & \courses \neq \COURSE
  }{}{
    \actOPCrsUpd: & \courses \bcmsuch \courses \subset
    \courses^\prime \land \card(\courses^\prime) \leq \maxC
  }
  $
  \Bvspace
  $
  \event[anticipated]{\CloseCourses}{}{\cs}{
    \Bgrd{grd0\_1}: & \cs \subseteq \courses \\
    \Bgrd{grd0\_2}: & \cs \neq \emptyset
  }{}
  {
    \Bact{act0\_1}: & \courses \bcmeq \courses \setminus \cs
  }
  $
\end{Bcode}

We choose purposely to model these events using different features of
\eventB.  In $\OpenCourses$, we use a nondeterministic action to model
the fact that some new courses are opened, i.e., $\courses \subset
\courses^\prime$, as long as the number of opened courses will not
exceed its limit, i.e., $\card(\courses^\prime) \leq \maxC$.  The guard
of the event states that the current number of opened courses has not
yet reached the limit.

$\CloseCourses$ models the set of courses that are going to be closed
using parameter $\cs$.  It is a non-empty set of currently opened
courses which is captured by $\CloseCourses$' guard.  The action is
modelled straightforwardly by removing $\cs$ out of the set
$\courses$.

We set the convergence status for $\OpenCourses$ and $\CloseCourses$
to be \emph{ordinary} and \emph{anticipated}, respectively.  We delay
the reasoning about the convergence of $\CloseCourses$ to later
refinements.  Our intention is to prove that there can be only
finitely many occurrences of $\CloseCourses$ between any two
$\OpenCourses$ events.

We present some of the obligations to illustrate what needs to be
proved for the consistency of $\coursesMch$.  We applied the proof
obligation rules as showed earlier in this Section.  Notice that we
can take the axioms and theorems of the seen context $\coursesCtx$ as
hypotheses in the proof obligations.  For clarity, we show only parts
of the hypotheses that are relevant for discharging the proof
obligations.  Moreover, we also show the proof obligations in their
simplified form, e.g., when the events' assignments are deterministic.

\paragraph{$\thmpo{\thmFinCrs}$} This obligation is in order to
ensure that $\thmFinCrs$ is derivable from previously declared
invariants.
\begin{Bcode}
  $\ldots$
\end{Bcode}
The proof obligation holds trivially since $\courses$ is a subset of a
finite set, i.e., $\COURSE$.

\paragraph{$\invpo{\init}{\invCrsLim}$} This obligation ensures that the
initialisation $\init$ establishes invariant $\invCrsLim$.
\begin{Bcode}
  $\ldots$
\end{Bcode}
Since the cardinality of the empty set $\emptyset$ is $0$, the proof
obligation holds trivially. 

\paragraph{$\grdthmpo{\OpenCourses}{\thmCrsBelowLim}$}  This obligation
ensures that $\thmCrsBelowLim$ is derivable from the invariants and the
previously declared guards of $\OpenCourses$. 
\begin{Bcode}
  $
  \ldots
  $
\end{Bcode}
Informally, we can derive from the hypotheses that $\card(\courses) <
\card(\COURSE)$, hence $\courses$ must be different from $\COURSE$.

\paragraph{$\fispo{\OpenCourses}{\actOPCrsUpd}$} This obligation
ensures that the nondeterministic assignment of $\OpenCourses$ is
feasible when the event is enabled.
\begin{Bcode}
  $
  \ldots
  $
\end{Bcode}
The reasoning about the proof obligation is as follows.  Since
$\courses$ is different from $\COURSE$, there exists an element
$\course$ which is closed, i.e., not in $\courses$.  By adding $\course$
to the set of opened courses, we strictly increase the number of
opened courses by $1$.  Moreover, the number of opened courses after
executing the event is still within the limit since originally it is
strictly below the limit.

\paragraph{$\invpo{\CloseCourses}{\invCrsLim}$}
This obligation is simplified accordingly since the assignment is
deterministic.  The purpose of the obligation is to prove that
$\CloseCourses$ maintains invariant $\invCrsLim$.
\begin{Bcode}
  $
  \ldots
  $
\end{Bcode}
Since removing some courses $\cs$ from the set of opened courses
$\courses$ can only reduce its number, the proof obligation can be
trivially discharged.

\paragraph{$\thmpo{\thmDlf}$}
The deadlock-freeness condition is encoded as theorem $\thmDlf$ of 
machine $\coursesMch$, which results in the following proof
obligation.
\begin{Bcode}
  $
  \ldots
  $
\end{Bcode}
We reason as follows.  If $\card(\courses) \neq \maxC$, the goal
trivially holds.  Otherwise, i.e., $\card(\courses) = \maxC$, since
$\maxC \neq 0$, we have that $\courses \neq \emptyset$.  As a result,
we can prove that $\exists \cs \qdot \cs \subseteq \courses \land \cs
\neq \emptyset$ by instantiating $\cs$ with $\courses$ itself.

\section{Context Extension}
\label{sec:context-extension}

\newBctx[membersCtx]{membersCtx}
\newBset[MEMBER]{MEM}
\newBaxm[axmFinMem]{axm1\_1}
\subsection{Context $\membersCtx$}

This is an initial context (i.e., does not extend any other context)
containing a carrier sets $\MEMBER$.  $\MEMBER$ represents the set of
club members, with an axiom stating that it is finite.
\begin{Bcode}
  $
  \carriersets{\MEMBER}
  $
  \Bhspace
  $
  \axioms{
    \axmFinMem: & \finite(\MEMBER)
  }
  $
\end{Bcode}


\newBctx[participantsCtx]{participantsCtx}
\newBcst[PARTICIPANT]{PRTCPT}
\newBaxm[axmPcpTyp]{axm1\_2}
\newBthm[thmFinPcp]{thm1\_1}
\subsection{Context $\participantsCtx$}
This context extends the previously defined context $\membersCtx$ and
is as follows.
\begin{Bcode}
  $
  \constants{\PARTICIPANT}
  $
  \Bhspace
  $
  \axioms{
    \axmPcpTyp: & \PARTICIPANT \subseteq \MEMBER \\
    \thmFinPcp: & \finite(\PARTICIPANT)
  }
  $
\end{Bcode}
Constant $\PARTICIPANT$ denotes the set of participants which must be
members of the club ($\axmPcpTyp$).  Theorem $\thmFinPcp$ states that
there can be only a finite number of participants, which gives rise to
the following trivial proof obligation.
\begin{Bcode}
  $
  \ldots
  $
\end{Bcode}
An important point is that axiom $\axmFinMem$ of the abstract context
$\membersCtx$ appears as a hypothesis in the proof obligation.

\newBctx[instructorsCtx]{instructorsCtx}
\newBcst[INSTRUCTOR]{INSTR}
\newBcst[instructors]{instrs}
\newBaxm[axmITRTyp]{axm1\_3}
\newBaxm[axmItrTyp]{axm1\_4}
\subsection{Context $\instructorsCtx$}
This context extends two contexts $\coursesCtx$ and $\membersCtx$, and
introduces two constants, namely $\INSTRUCTOR$ and $\instructors$.
$\INSTRUCTOR$ models the set of instructors which are members of the
club ($\Baxm{axm1\_3}$).  Constant $\instructors$ models the
relationship between courses and instructors and is constrained by
$\Baxm{axm1\_4}$: it is a \emph{total function} from $\COURSE$ to
$\INSTRUCTOR$.
\begin{Bcode}
  $
  \constants{\INSTRUCTOR, \instructors}
  $
  \Bvspace
  $
  \axioms{
    \axmITRTyp: & \INSTRUCTOR \subseteq \MEMBER \\
    \axmItrTyp: & \instructors \in \COURSE \tfun \INSTRUCTOR
  }
  $
\end{Bcode}

\section{Machine Refinement}
\label{sec:machine-refinement}


\newBmch[participantsMch]{m1}
\newBvrb[participants]{prtcpts}
\newBinv[invPcpTyp]{inv1\_1}
\newBinv[invIntPcp]{inv1\_2}
\newBevt{Register}
\newBpar[participant]{p}
\subsubsection{Machine $\participantsMch$}
Machine $\participantsMch$ sees contexts $\instructorsCtx$ and
$\participantsCtx$.  As a result, it implicitly sees $\coursesCtx$ and
$\membersCtx$.  Variable $\courses$ is retained in this refinement.
An additional variable $\participants$ representing information about
course participants is introduced.  Invariant $\Binv{inv1\_1}$ models
$\participants$ as a relation between the set of opened courses
$\courses$ and the set of participants $\PARTICIPANT$.  Invariant
$\Binv{inv1\_2}$ states that for every opened course $\course$, the
instructor of that course, i.e., $\instructors(\course)$, is not
amongst its participants, represented by $\participants[\{\course\}]$.
\begin{Bcode}
  $
  \variables{\courses, \participants}
  $
  \Bvspace
  $
  \invariants{
    \invPcpTyp: & \participants \in \courses \rel \PARTICIPANT \\
    \invIntPcp: & \forall \course ~\qdot~ \course \in \courses ~\limp~
    \instructors(\course) \notin \participants[\{\course\}]
  }
  $
  \Bvspace
  $
  \initialisation{
    \ldots \\
    \participants \bcmeq \emptyset
  }
  $
\end{Bcode}
Initially, there are no opened courses hence $\participants$ is
assigned to be $\emptyset$.
The original abstract event $\OpenCourses$ stays unchanged in this
refinement, while an additional assignment is added to $\CloseCourses$
to update $\participants$ by removing the information about
the set of closing courses $\cs$ from it.
\begin{Bcode}
  $ \event[anticipated]{\CloseCourses}{}{\cs}{\ldots}{}{
    \ldots \\
    \Bact{act1\_2}: & \participants \bcmeq \cs \domsub \participants }
  $
\end{Bcode}

A new event is added, namely $\Register$, to model the registration of
a participant $\participant$ for an opened course $\course$.  The
guard of the event ensures that $\participant$ is not the instructor
of the course ($\Bgrd{grd1\_3}$) and is not yet registered for the
course ($\Bgrd{grd1\_4}$).  The action of the event update
$\participants$ accordingly by adding the mapping $\course
\mapsto \participant$ to it.
\begin{Bcode}
  $
  \event[convergent]{\Register}{}{\participant, \course}{
    \Bgrd{grd1\_1}: & \participant \in \PARTICIPANT \\
    \Bgrd{grd1\_2}: & \course \in \courses \\
    \Bgrd{grd1\_3}: & \participant \neq \instructors(\course) \\
    \Bgrd{grd1\_4}: & \course \mapsto \participant
    \notin \participants
  }{}
  {
    \Bact{act1\_1}: & \participants \bcmeq \participants \bunion \{\course \mapsto \participant\}
  }
  $
\end{Bcode}

We attempt to prove that $\Register$ is convergent and $\CloseCourses$
is anticipated using the following variant.
\begin{Bcode}
  $
  \variant{(\courses \cprod \PARTICIPANT) ~\setminus~ \participants}
  $
\end{Bcode}
The variant is a set of mappings, each links an opened course to a
participant who has \emph{not} registered for the respective course.

We present some of the important proof obligations for
$\participantsMch$.  For events $\OpenCourses$ and $\CloseCourses$,
proof obligations are trivial.

\paragraph{$\invpo{\CloseCourses}{\invIntPcp}$} This obligation is to
ensure that $\invIntPcp$ is maintained by $\CloseCourses$.  The
obligation is trivial, in particular, given that $\course \notin \cs$,
we have $(\cs \domsub \participants)[\{\course\}]$ is the same as
$\participants[\{\course\}]$.
\begin{Bcode}
  $
  \ldots
  $
\end{Bcode}

\paragraph{$\invpo{\Register}{\invPcpTyp}$} This obligation is to
guarantee that $\invPcpTyp$ is maintained by the new event
$\Register$.
\begin{Bcode}
  $
  \ldots
  $
\end{Bcode}

\paragraph{$\finpo$} This obligation is to ensure that the declared
variant used for proving convergence of events is finite.  This is
trivial, since the set of opened course $\courses$ and the set of
participants $\PARTICIPANT$ are both finite.
\begin{Bcode}
  $
  \ldots
  $
\end{Bcode}

\paragraph{$\varpo{\CloseCourses}$} This proof obligation ensures that
anticipated event $\CloseCourses$ does not increase the variant.
\begin{Bcode}
  $
  \ldots
  $
\end{Bcode}

\paragraph{$\varpo{\Register}$} This proof obligation ensures that the
convergent event $\Register$ decreases the variant.  This is trivial
since a new mapping $\course \mapsto \participant$ is added to
$\participants$, effectively increasing $\participants$, hence
strictly decreasing the variant.
\begin{Bcode}
  $
  \ldots
  $
\end{Bcode}

\subsection{Machine $\datarefMch$}
We perform a data refinement by replacing abstract variables
$\courses$ and $\participants$ by a new concrete variable
$\attendees$.  This machine does not explicitly model any
requirements: it implicitly inherits requirements from previous
abstract machines.  As stated in invariant $\Binv{inv2\_1}$,
$\attendees$ is a \emph{partial function} from $\COURSE$ to some set
of participants (i.e., member of $\pow(\PARTICIPANT)$).  Invariants
$\Binv{inv2\_2}$ and $\Binv{inv2\_3}$ act as gluing invariants,
linking abstract variables $\courses$ and $\participants$ with
concrete variable $\attendees$.  Invariant $\Binv{inv2\_2}$ specifies
that $\courses$ is the domain $\attendees$.  Invariant
$\Binv{inv2\_3}$ states that for every opened courses $\course$, the
set of participants attending that course represented abstractly as
$\participants[\{\course\}]$ is the same as $\attendees(\course)$.
\begin{Bcode}
  $
  \variables{\attendees}
  $
  \Bvspace
  $
  \invariants{
    \Binv{inv2\_1}: & \attendees \in \COURSE \pfun \pow(\PARTICIPANT)
    \\
    \Binv{inv2\_2}: & \courses = \dom(\attendees) \\
    \Binv{inv2\_3}: & \forall \course \qdot \course \in \courses
    \limp \participants[\{\course\}] = \attendees(\course) \\
    \Bthm{thm2\_1}: & \finite(\attendees) \\[2ex]
  }
  $
  \Bvspace
  $
  \initialisation{
    \attendees \bcmeq \emptyset
  }
  $
\end{Bcode}
We illustrate our data refinement by the following example.  Assume
that the available courses $\COURSE$ are $\{\course_1, \course_2,
\course_3\}$, with $\course_1$ and $\course_2$ being opened, i.e.,
$\courses = \{\course_1, \course_2\}$.  Assume that $\course_1$ has no
participants, and $\participant_1$ and $\participant_2$ are attending
$\course_2$.  Abstract variable $\participants$ hence contains two
mappings as follows $\{\course_2 \mapsto \participant_1, \course_2
\mapsto \participant_2\}$.  The same information can be represented by
the concrete variable $\attendees$ as follows $\{\course_1 \mapsto
\emptyset, \course_2 \mapsto \{\participant_1, \participant_2\}\}$.

We refine the events using data refinement as follows.  Event
$\OpenCourses$ is refined by $\OpenCourse$ where one course (instead
of possibly several courses) is opened at a time.  The course that is
opening is represented by the concrete parameter $\course$.
\begin{Bcode}[\footnotesize]
  $
  \event{\OpenCourses}{}{}{\Bgrd{grd0\_1}: & \card(\courses)
    \neq \maxC \\
    \Bthm{thm0\_1}: & \courses \neq \COURSE
  }{}{
    \Bact{act0\_1}: & \courses \bcmsuch
    \courses \subset \courses^\prime \land 
    \card(\courses^\prime) \leq \maxC
  }
  $
  \Bhspace[2em]
  $
  \event{\OpenCourse}{\OpenCourses}{\course}{
    \Bgrd{grd2\_1}: & \course \notin \dom(\attendees) \\
    \Bgrd{grd2\_2}: & \card(\attendees) \neq \maxC
  }{
    \courses^\prime = \courses \bunion \{\course\}
  }{
    \Bact{act2\_1}: & \attendees(\course) \bcmeq \emptyset
  }
  $
\end{Bcode}
The concrete guards ensure that $\course$ is a closed course and the
number of opened course ($\card(\attendees)$) has not reached the
limit $\maxC$.  The action of $\OpenCourse$ sets the initial
participants for the newly opened course $\course$ to be the empty
set.  In order to prove the refinement relationship between
$\OpenCourse$ and $\OpenCourses$, we need to give the witness for the
after value of the disappearing variable $\courses^\prime$.  In this
case, it is specified as $\courses^\prime = \courses \bunion
\{\course\}$, i.e., adding the newly opened course $\course$ to the
original set of opened courses $\courses$.

Abstract event $\CloseCourses$ is refined by concrete event
$\CloseCourse$, where one course $\course$ (instead of possibly
several courses $\cs$) is closed at a time.  The guard and action of
concrete event $\CloseCourse$ are as expected.
\begin{Bcode}
  $
  \event[anticipated]{\CloseCourses}{}{\cs}{
    \Bgrd{grd0\_1}: & \cs \subseteq \courses \\
    \Bgrd{grd0\_2}: & \cs \neq \emptyset
  }{}
  {
    \Bact{act0\_1}: & \courses \bcmeq \courses \setminus \cs \\
    \Bact{act2}: & \participants \bcmeq \cs \domsub \participants
  }
  $
  \Bhspace
  $
  \event[convergent]{\CloseCourse}{\CloseCourses}{\course}{
    \Bgrd{grd2\_1}: & \course \in \dom(\attendees)
  }{
    \cs = \{\course\}
  }
  {
    \Bact{act2\_1}: & \attendees \bcmeq \{\course\} \domsub \attendees
  }
  $
\end{Bcode}
We need to give the witness for the disappearing abstract parameter
$\cs$. It is specified straightforwardly as $\cs = \{\course\}$.
Notice also that we change the convergence status of $\CloseCourse$
from \emph{anticipated} to \emph{convergent}.  We use the following
variant to prove that $\CloseCourse$ is convergent.
\begin{Bcode}
  $
  \variant{\card(\attendees)}
  $
\end{Bcode}
The variant represents the number of mappings in $\attendees$, and
since it is a partial function, it is also the same as the number of
elements in its domain, i.e., $\card(\attendees) =
\card(\dom(\attendees))$.  As a result, the variant represent the
number of opened courses.

Event $\Register$ is refined as follows%
\footnote{%
  We use prefixes $\Bevt{(abs\_)}$ and $\Bevt{(cnc\_)}$ to denote the
  abstract and concrete version of the event, accordingly.%
}, such that references to $\courses$ and $\participants$ in guard and
action are replaced by references to $\attendees$.
\begin{Bcode}[\scriptsize]
  $
  \event{(abs\_)\Register}{}{\participant, \course}{
    \Bgrd{grd1\_1}: & \participant \in \PARTICIPANT \\
    \Bgrd{grd1\_2}: & \course \in \courses \\
    \Bgrd{grd1\_3}: & \participant \neq \instructors(\course) \\
    \Bgrd{grd1\_4}: & \course \mapsto \participant
    \notin \participants
  }{}
  {
    \Bact{act1\_1}: & \participants \bcmeq \participants \bunion \{\course \mapsto \participant\}
  }
  $
  \hspace{2em}
  $
  \event{(cnc\_)\Register}{\Register}{\participant, \course}{
    \Bgrd{grd2\_1}: & \participant \in \PARTICIPANT \\
    \Bgrd{grd2\_2}: & \course \in \dom(attendees) \\
    \Bgrd{grd2\_3}: & \participant \neq \instructors(\course) \\
    \Bgrd{grd2\_4}: & \participant \notin \attendees(\course)
  }
  {}
  {
    \Bact{act2\_1}: & \attendees(\course) \bcmeq \attendees(\course)
    \bunion \{\participant\}
  }
  $
\end{Bcode}

We now show some proof obligations for proving the refinement of
$\participantsMch$ by $\datarefMch$.

\paragraph{$\simpo{\OpenCourse}{act0\_1}$} This proof obligation
ensures that the action $\Bact{act0\_1}$ of abstract event
$\OpenCourses$ can simulate the action of concrete event
$\OpenCourse$.  Notice the use of the witness for $\courses^\prime$ as
a hypothesis in the obligation.
\begin{Bcode}
  $
  \ldots
  $
\end{Bcode}

\paragraph{$\grdpo{CloseCourse}{grd0\_1}$} This proof obligation
ensures that the guard of concrete event $\CloseCourse$ is stronger
than the abstract guard $\Bgrd{grd0\_1}$ of abstract event
$\CloseCourses$.  Note the use of the witness for $\cs$ as a
hypothesis in the obligation.
\begin{Bcode}
  $
  \ldots
  $
\end{Bcode}

\paragraph{$\natpo{\CloseCourse}$} This proof obligation ensures that
the variant is a natural number when $\CloseCourse$ is enabled.
\begin{Bcode}
  $
  \ldots
  $
\end{Bcode}

\paragraph{$\varpo{\CloseCourse}$} This proof obligation ensures that
the variant is strictly decreased by $\CloseCourse$.  The obligation
is trivial since the variant represents the number of opened courses
and $\CloseCourse$ closes one of them.
\begin{Bcode}
  $
  \ldots
  $
\end{Bcode}

\end{document}